\documentclass{article}
\usepackage[utf8]{inputenc}
\usepackage{amsmath}
\usepackage{amsfonts}
\usepackage{amsthm}
\usepackage{amssymb}
\usepackage{mathtools}
\usepackage{physics}
\usepackage[left=2.5cm,right=2.5cm,top=1.6cm,bottom=1.6cm]{geometry}
\usepackage{cancel}
\usepackage{mathtools}
\usepackage{ulem}
\usepackage{polynom}
\usepackage{comment}
\usepackage{caption}
\def\ZZ{{\mathbb Z}}
\def\RR{{\mathbb R}}
\def\CC{{\mathbb C}}
\def\QQ{{\mathbb Q}}
\def\EE{{\mathbb E}}
\def\NN{{\mathbb N}}
\def\FF{{\mathbb F}}
\def\AA{{\mathcal A}}
\def\SS{{\mathbb S}}
\def\PP{{\mathbb P}}
\def\HH{{\mathbb H}}
\def\sp{{\operatorname{span}}}
\def\nul{{\operatorname{nullity}}}
\def\nu{{\operator{Null}}}
\def\ran{{\operatorname{ran}}}
\def\Int{{\operatorname{Int}}}
\def\momt{{-i\hbar \frac{d}{dx}}}
\def\hamt{{-\frac{\hbar^2}{2m} \frac{d^2}{dx^2}+V(x)}}
\DeclarePairedDelimiter\ceil{\lceil}{\rceil}
\DeclarePairedDelimiter\floor{\lfloor}{\rfloor}
\begin{document}
\begin{titlepage}
    \centering
    \vspace*{\fill}

    \vspace*{0.5cm}

    \huge\bfseries
    AMATH 251 Project 1: Numerical methods for solving ODEs

    \vspace*{0.5cm}

    \large Yu Li \& Lev Raizman \\
    October 2021

    \vspace*{\fill}

\end{titlepage}
\section{Quadratic Approximation}
The quadratic approximation method is given by
\begin{equation}
    y_{j+1} =  y(x_j) +y'(x_j)h + y''(x_j) \frac{h^2}{2}
\end{equation}
To determine the order of this method,
we assume the solution \(y\) to the differential equation is at least \(C^3\). Then, by the Taylor’s Approximation Theorem,
we know that \(y(x_{j+1})\) can be written as the following form:
\begin{equation}
  y(x_{j+1}) = y(x_j +h) = y(x_j) +y'(x_j)h + y''(x_j) \frac{h^2}{2} + O(h^3)
\end{equation}
when \(h\to 0\).
Then, we can see the truncation error \(E\) for each step of the quadratic approximation satisfies:
\begin{equation}
  \lim_{h\to 0}\frac{E}{h^3} =\lim_{h\to 0} \frac{y(x_{j+1}) - y_{j+1}}{h^3} = \lim_{h\to 0}\frac{  y(x_{j+1}) - ( y(x_j) +y'(x_j)h + y''(x_j) \frac{h^2}{2})}{h^3} = \lim_{h\to 0} \frac{O(h^3)}{h^3} = \text{const}
\end{equation}
where in general \(\text{const}\) is non-zero. This shows the truncation error for each step of the quadratic approximation
is \(O(h^3)\). When using the step size \(h \propto \frac{1}{N}\), where \(N\) is the number of steps, we have \(N \propto \frac{1}{h}\). Hence, the global error is:
\begin{equation}
  E_{global} = N E  = \frac{1}{h} O(h^3) = O(h^2)
\end{equation}
This shows the quadratic approximation is a second-order method. \\
\\
I have used Maple to write a program that approximates the solutions of first-order DEs by using the quadratic approximation.
See part 1 of the appendix for the code and commands. Note that in the appendix, the IVP \(y' = xy\) with \(y(0)=1\) is used
to demonstrate the functionality of the code. The step size \(h\) is chosen to be \(0.00025\).
So, the number of steps is \(N = \frac{1-0}{0.00025} = 4000\). The numerical method yields the following result:
\begin{equation}
  y(1) = 1.648721184
\end{equation}
We can quickly verify that the actual analytical solution to this IVP is
\begin{equation}
  y = e^{x^2/2}
\end{equation}
by substituting this solution into the IVP. So, we know that
\begin{equation}
  y(1) = e^{1/2} \approx 1.648721271
\end{equation}
As we can see, the result of the numerical method is pretty close to the actual value. \\
\\
We remark that users can modify the corresponding sections in the code to use the program for solving an arbitrary first-order DE.\\
\\
To compare different numerical methods (Euler, RK2 and quadratic) for solving first-order DEs, we use the same IVP,
\(y' = xy\) with \(y(0)=1\), as above. The number of steps are chosen to be \(1,2,4\) and \(8\). Since
\(h   = \frac{x_{\max} - x_0}{N} = \frac{1-0}{N} = \frac{1}{N}\) (\(N\) is the number of steps),
we know that the step sizes \(h\) are \(1, 0.5, 0.25\) and \(0.125\) respectively.
We enter these step sizes into Maple to produce the approximate values of \(y(1)\) by using these three methods.
See the part 1 of the appendix for the code. The results are presented in the following table:
\begin{center}
\begin{tabular}{||c |c| c| c||}
 \hline
  Number of steps \(N\) & \(y(1)\) (Euler's Method) & \(y(1)\) (RK2) & \(y(1)\) (Quadratic Approximation) \\ [0.5ex]
 \hline\hline
 1 & 1 & 1.5 & 1.5 \\
 \hline
 2 & 1.25 & 1.599609375 & 1.582031250 \\
 \hline
 4 & 1.419433594 & 1.634217274 & 1.626173613 \\
 \hline
 8 & 1.524005973 & 1.644771474 & 1.642191679\\
 \hline
\end{tabular}
\captionof{table}{The approximate values of \(y(1)\) to the IVP \(y' = xy\) with \(y(0)=1\) by using different numerical methods and
different numbers of steps.}
\end{center}
Compared to the actual value \(y(1) = e^{1/2} \approx 1.648721271\), we see that the RK2 and the quadratic approximation
produce much less error than the Euler's method. And it is noticeable that the errors of the RK2 and the quadratic approximation
are approximately the same. As we will see in the next section, both the RK2 and the quadratic approximation have the same global
error of \(O(h^2)\).
\section{The Second-order Runga-Kutta Method (RK2)}
The second-order Runga-Kutta method (RK2) is given by the following:
\begin{equation}
  k_1 = hf(x_n, y_n)
\end{equation}
\begin{equation}
  k_2 = hf( x_n+ \frac{1}{2}h, y_n + \frac{1}{2}k_1)
\end{equation}
\begin{equation}
  y_{n+1} = y_n +k_2
\end{equation}
We show that the RK2 also gives an error of \(O(h^3)\) at each step. To show this, we assume the solution \(y\) to the differential
equation is at least \(C^3\) and \(f(x,y)\) is at least \(C^2\). By the Taylor's Approximation Theorem for multivariable
functions, we know that:
\begin{align}
  &f( x_n+ \frac{1}{2}h, y_n + \frac{1}{2}k_1) = f(x_n, y_n) + \frac{h}{2}\frac{\partial f}{\partial x}(x_n, y_n) +
  \frac{k_1}{2} \frac{\partial f}{\partial y}(x_n, y_n) + O(h^2)\\
  &=  f(x_n, y_n) + \frac{h}{2}\frac{\partial f}{\partial x}(x_n, y_n) +
  \frac{h}{2} f(x_n, y_n)\frac{\partial f}{\partial y}(x_n, y_n) + O(h^2)
\end{align}
as \(h\to 0\).
Here we have substituted the expression of \(k_1\) given by equation (8).
From the differential equation \(y' = f(x,y)\), we have \(y'(x_n) = f(x_n, y_n)\). Substitute this into the above
equation, we have:
\begin{equation}
  f( x_n+ \frac{1}{2}h, y_n + \frac{1}{2}k_1) = y'(x_n) + \frac{h}{2} \left(\frac{\partial f}{\partial x}(x_n, y_n) +
  y'(x_n)\frac{\partial f}{\partial y}(x_n, y_n) \right) +O(h^2)
\end{equation}
By the multivariable chain rule, we know that
\begin{equation}
  \frac{\dd f}{\dd x} (x_n, y_n) = \frac{\partial f}{\partial x}(x_n, y_n) + y'(x_n)\frac{\partial f}{\partial y}(x_n, y_n)
\end{equation}
Differentiate both sides of \(y'(x_n) = f(x_n, y_n)\) with respect to \(x\), we have \(y''(x_n) = \frac{\dd f}{\dd x}(x_n, y_n)\).
Therefore:
\begin{equation}
  y''(x_n) = \frac{\partial f}{\partial x}(x_n, y_n) + y'(x_n)\frac{\partial f}{\partial y}(x_n, y_n)
\end{equation}
Substitute this into equation (13), we get:
\begin{equation}
  f( x_n+ \frac{1}{2}h, y_n + \frac{1}{2}k_1)  = y'(x_n) + \frac{h}{2} y''(x_n)  +O(h^2)
\end{equation}
So:
\begin{equation}
  k_2 = h f( x_n+ \frac{1}{2}h, y_n + \frac{1}{2}k_1)  = y'(x_n)h + \frac{1}{2} y''(x_n) h^2 + O(h^3)
\end{equation}
\begin{equation}
  y_{n+1} = y_n +k_2 = y(x_n) +  y'(x_n)h + \frac{1}{2} y''(x_n) h^2 + O(h^3)
\end{equation}
Compared to equation (2) (identifying the index \(j\) in equation (2) as \(n\) here), we see that
the truncation error \(E\) for each step of the RK2 methods satisfies:
\begin{align}
    \lim_{h\to 0} \frac{E}{h^3}
    & = \lim_{h\to 0} \frac{y(x_{n+1}) - y_{n+1}}{h^3}\\
    & = \lim_{h\to 0} \frac{y(x_n) +  y'(x_n)h + \frac{1}{2} y''(x_n) h^2 + O(h^3) - (y(x_n) +  y'(x_n)h + \frac{1}{2} y''(x_n) h^2 + O(h^3))}{h^3}\\
    &= \lim_{h\to 0} \frac{O(h^3)}{h^3} = \text{const}
\end{align}
where in general \(\text{const}\) is non-zero. This shows the truncation error for each step of the RK2 method
is \(O(h^3)\). When using the step size \(h \propto \frac{1}{N}\), where \(N\) is the number of steps, we have \(N \propto \frac{1}{h}\). Hence, the global error is:
\begin{equation}
  E_{global} = N E  = \frac{1}{h} O(h^3) = O(h^2)
\end{equation}
This shows the RK2 method is also a second-order method.\\ \\  \\  \\ \\ \\
\\
\section{Two Exampler Initial Value Problems}
\subsection{Problem 1}
\begin{eqnarray}
  y'= x^{3/2} y & \text{with} & y(0)=1
\end{eqnarray}
For comparison, we find the analytical solution first. Clearly, this differential equation is separable:
\begin{equation}
  \int \frac{\dd y}{y} = \int x^{3/2} \dd x
\end{equation}
\begin{equation}
  \ln|y| = \frac{2}{5}x^{5/2} + C
\end{equation}
\begin{equation}
   y = \pm \exp(\frac{2}{5}x^{5/2} + C) = \pm e^C \exp(\frac{2}{5}x^{5/2}) = A \exp(\frac{2}{5}x^{5/2})
\end{equation}
where \(A = \pm e^C\) is an arbitrary non-zero constant. From the initial condition \(y(0)=1\), we have:
\begin{equation}
  1 = A \exp(\frac{2}{5}\cdot 0^{5/2}) = A
\end{equation}
Therefore, the analytical solution to this IVP is:
\begin{equation}
   y = \exp(\frac{2}{5}x^{5/2})
\end{equation}
The value of \(y\) at \(x= 1\) is:
\begin{equation}
  y = e^{2/5} \approx 1.49182469764
\end{equation}
\\
I have used Maple to approximate the value of \(y(1)\) by using Euler's method, the quadratic approximation method and the
RK2 method. The step sizes are chosen to be \(2^n\) where \(n = 0,1,2,\ldots, 10\). The results are presented in the
following table:
\begin{center}
\begin{tabular}{||c |c| c| c||}
 \hline
  Number of steps \(N\) & \(y(2)\) (Euler's Method) & \(y(2)\) (RK2) & \(y(2)\) (Quadratic Approximation) \\ [0.5ex]
 \hline\hline
 \(2^0\)  & 1. & 1.353553391 & 1. \\
 \hline
 \(2^1\)  & 1.176776695 & 1.438056014 & 1.324984216 \\
 \hline
 \(2^2\)  & 1.304655609 & 1.475269449 & 1.444233307 \\
 \hline
 \(2^3\)  & 1.388135354 & 1.487255639 & 1.479275953 \\
 \hline
 \(2^4\)  & 1.436972160 & 1.490625573 & 1.488634055 \\
 \hline
 \(2^5\)  & 1.463571977 & 1.491517358 & 1.491025904 \\
 \hline
 \(2^6\)  & 1.477481149 & 1.491746838 & 1.491625855 \\
 \hline
 \(2^7\)  & 1.484597190 & 1.491805082 & 1.491775268 \\
 \hline
 \(2^8\)  & 1.488196810 & 1.491819775 & 1.491812404 \\
 \hline
 \(2^9\)  & 1.490007202 & 1.491823478 & 1.491821633 \\
 \hline
 \(2^{10}\) & 1.490915038 & 1.491824383 & 1.491823934 \\
 \hline
\end{tabular}
\captionof{table}{The approximate values of \(y(1)\) to the IVP \(y' = x^{3/2}y\) with \(y(0)=1\) by using different numerical methods and different numbers of steps.}
\end{center}
By using Maple, I have calculated the error \(E\) of the approximate values of \(y(1)\) compared to the actual value. If the error \(E\propto h^{\alpha}\), then \(\log_b|E| =C + \alpha \log_b h\) for some constants \(C\) and \(\alpha\).
So, in order to verify that the Euler's method is first-order and the RK2 and the quadratic approximation are second-order, we can
plot the error versus \(h\) plot on a log-log plot and show that the slope of the curves are approximately 1 and 2 for small enough \(h\) (when \(h\) is large, the error will not be exactly proportional to \(h^\alpha\)).
The plot has been generated by using Maple. See the Appendix for the data and the code.
\pagebreak \\ \\
\begin{figure}
    \centering
    \includegraphics[scale=0.5]{q3a.PNG}
    \caption{The Error versus \(h\) plot on a log-log plot. The base is chosen to be \(e\). Several reference curves are also given. }
    \label{fig:label1}
\end{figure}\text{ }\\
Several reference curves of slope 1 and 2 are also included in the plot. As we can see on the plot, when \(h\) is small, the
slope of the error versus \(h\) plot of the Euler's method is pretty close to the reference curve with slope 1;
the slopes of the error versus \(h\) plots of the RK2 method and quadratic approximation are pretty close to the reference curves
with slope 2. By the above argument, this verifies the Euler's method is first-order and the RK2 and the quadratic approximation are second-order.
\pagebreak \\ \\
\subsection{Problem 2}
\begin{eqnarray}
  y' = 2y - 2e^{-x/2} +\cos(2x) -2\sin(8x) & \text{with} & y(0)=1
\end{eqnarray}
For comparison, we find the analytical solution first. This is a first-order linear DE. The corresponding homogeneous solution is:
\begin{equation}
  y' -2y = 0
\end{equation}
Therefore:
\begin{equation}
  \int \frac{\dd y}{y} = 2\int\dd x
\end{equation}
\begin{equation}
  \ln |y| = 2x +C
\end{equation}
\begin{equation}
  y = \pm e^C e^{2x} = A e^{2x}
\end{equation}
where \(A\) is an arbitrary non-zero constant. Since any anti-derivative works, we can take
\begin{equation}
  y_h = e^{2x}
\end{equation}
as a homogeneous solution. \\
\\
Let \(y_1 = v_1 y_h\) be a particular solution to \(y'-2y = -2e^{-x/2}\). Therefore, we have:
\begin{equation}
  (v_1y_h)' -2(v_1 y_h) = v_1'y_h + v_1y_h' -2v_1y_h = v_1'y_h - v_1(y_h' - 2y_h) = v_1'y_h = -2e^{-x/2}
\end{equation}
\begin{equation}
  v_1' = -2e^{-x/2}/ y_h= -2e^{-x/2}/ e^{2x} = -2e^{-5x/2}
\end{equation}
\begin{equation}
  v_1 = -2\int e^{-5x/2} \dd x = \frac{4}{5} e^{-5x/2} + C
\end{equation}
Since any anti-derivative works, we can take:
\begin{equation}
  v_1 = \frac{4}{5} e^{-5x/2}
\end{equation}
So:
\begin{equation}
  y_1 = y_h v_1 = \frac{4}{5} e^{-x/2}
\end{equation}
Let \(y_2 = v_2y_h\) be a particular solution to \(y'-2y = \cos(2x)\). Therefore:
\begin{equation}
  (v_2y_h)' -2(v_2 y_h) = v_2'y_h + v_2y_h' -2v_2y_h = v_2'y_h - v_2(y_h' - 2y_h) = v_2'y_h = \cos(2x)
\end{equation}
\begin{equation}
  v_2' = \cos(2x) / y_h = e^{-2x}\cos(2x)
\end{equation}
Using the integration identity \(\int e^{ax}\cos(bx) \dd x = \frac{1}{a^2+b^2}e^{ax} (a\cos bx + b\sin bx)+C\) from MATH 148, we have:
\begin{equation}
  v_2 = \int e^{-2x}\cos(2x) \dd x = \frac{1}{8}e^{-2x}(-2 \cos(2x) + 2\sin(2x)) +C = \frac{1}{4}e^{-2x}(\sin(2x)-\cos(2x)) +C
\end{equation}
Since any anti-derivative works, we take:
\begin{equation}
  v_2 = \frac{1}{4}e^{-2x}(\sin(2x)-\cos(2x))
\end{equation}
\begin{equation}
  y_2 =y_h v_2 = \frac{1}{4}(\sin(2x)-\cos(2x))
\end{equation}
Let \(y_3 = v_3 y_h\) be a particular solution to \(y' -2y = -2\sin(8x)\). Therefore:
\begin{equation}
  (v_3y_h)' -2(v_3 y_h) = v_3'y_h + v_3y_h' -2v_3y_h = v_3'y_h - v_3(y_h' - 2y_h) = v_3'y_h = -2\sin(8x)
\end{equation}
\begin{equation}
  v_3' = -2\sin(8x)/ y_h = -2 e^{-2x} \sin(8x)
\end{equation}
Using the integration identity \(\int e^{ax}\sin(bx) \dd x = \frac{1}{a^2+b^2}e^{ax} (a\sin bx - b\cos bx)+C\) from MATH 148, we have:
\begin{equation}
  v_3 = -2 \int e^{-2x} \sin(8x) \dd x = -2\left(\frac{1}{68}e^{-2x}(-2\sin(8x) - 8\cos(8x))\right) + C=
  \frac{1}{17}e^{-2x}(\sin(8x) + 4\cos(8x)) +C
\end{equation}
Since any anti-derivative works, we choose
\begin{equation}
  v_3 = \frac{1}{17}e^{-2x}(\sin(8x) + 4\cos(8x))
\end{equation}
\begin{equation}
  y_3 = y_h v_3 = \frac{1}{17}(\sin(8x) + 4\cos(8x))
\end{equation}
Therefore, by the linearity of the given differential equation, we know that the general solution is:
\begin{equation}
  y = Cy_h + y_1 + y_2 +y_3 = Ce^{2x} +  \frac{4}{5} e^{-x/2} + \frac{1}{4}(\sin(2x)-\cos(2x)) + \frac{1}{17}(\sin(8x) + 4\cos(8x))
\end{equation}
where \(C\) is an arbitrary constant. From the initial condition \(y(0)=1\), we have:
\begin{equation}
  1 = C + \frac{4}{5} - \frac{1}{4} + \frac{4}{17}
\end{equation}
\begin{equation}
  C = \frac{73}{340}
\end{equation}
So, the analytical solution to the given IVP is:
\begin{equation}
  y = \frac{73}{340}e^{2x} +  \frac{4}{5} e^{-x/2} + \frac{1}{4}(\sin(2x)-\cos(2x)) + \frac{1}{17}(\sin(8x) + 4\cos(8x))
\end{equation}
The value of \(y(2)\) is
\begin{equation}
  y(2) = \frac{73}{340}e^{4} + \frac{4}{5} e^{-1} + \frac{1}{4}(\sin(4)-\cos(4)) + \frac{1}{17}(\sin(16) + 4\cos(16))
  \approx 11.74879068
\end{equation}
\\
I have used Maple to approximate the value of \(y(2)\) by using Euler's method, the quadratic approximation method and the
RK2 method. The step sizes are chosen to be \(2^n\) where \(n = 0,1,2,\ldots, 10\). The results are presented in the
following table:
\begin{center}
\begin{tabular}{||c |c| c| c||}
 \hline
  Number of steps \(N\) & \(y(2)\) (Euler's Method) & \(y(2)\) (RK2) & \(y(2)\) (Quadratic Approximation) \\ [0.5ex]
 \hline\hline
 \(2^0\) & 3.0 & 1.784150702 & -23.0000 \\
 \hline
 \(2^1\) & 2.392075351 & 18.01202423 & -29.15788113 \\
 \hline
 \(2^2\) & 8.953896933 & 7.137321825 & -0.8887624689 \\
 \hline
 \(2^3\) & 8.905032042 & 10.04907103 & 6.746093247\\
 \hline
 \(2^4\) & 9.749530982 & 11.19097463 & 10.04027214 \\
 \hline
 \(2^5\) & 10.53169214 & 11.58658252 & 11.25346002 \\
 \hline
 \(2^6\) & 11.06961427 & 11.70496697 & 11.61605933 \\
 \hline
 \(2^7\) & 11.38872156 & 11.73739905 & 11.71448939 \\
 \hline
 \(2^8\) & 11.56321419 & 11.74588680 & 11.74007606 \\
 \hline
 \(2^9\) & 11.65455896 & 11.74805765 & 11.74659453 \\
 \hline
 \(2^{10}\) & 11.70130654 & 11.74860654 & 11.74823937 \\
 \hline
\end{tabular}
\captionof{table}{The approximate values of \(y(2)\) to the IVP \(y' = 2y - 2e^{-x/2} +\cos(2x) -2\sin(8x)\) with \(y(0)=1\) by using different numerical methods and different numbers of steps.}
\end{center}
By using Maple, I have calculated the error \(E\) of the approximate values of \(y(2)\) compared to the actual value. If the error \(E\propto h^{\alpha}\), then \(\log_b|E| =C + \alpha \log_b h\) for some constants \(C\) and \(\alpha\).
So, in order to verify that the Euler's method is first-order and the RK2 and the quadratic approximation are second-order, we can
plot the error versus \(h\) plot on a log-log plot and show that the slope of the curves are approximately 1 and 2 for small enough \(h\) (when \(h\) is large, the error will not be exactly proportional to \(h^\alpha\)).
The plot has been generated by using Maple. See the Appendix for the data and the code.
\pagebreak \\ \\
\begin{figure}
    \centering
    \includegraphics[scale=0.5]{q3b.PNG}
    \caption{The Error versus \(h\) plot on a log-log plot. The base is chosen to be \(e\). Several reference curves are also given. }
    \label{fig:label2}
\end{figure}\text{ }\\
Several reference curves of slope 1 and 2 are also included in the plot. As we can see on the plot, when \(h\) is small, the
slope of the error versus \(h\) plot of the Euler's method is pretty close to the reference curve with slope 1;
the slopes of the error versus \(h\) plots of the RK2 method and quadratic approximation are pretty close to the reference curves
with slope 2. By the above argument, this verifies the Euler's method is first-order and the RK2 and the quadratic approximation are second-order.
\pagebreak
\section{The Cost of the Improved Methods}
To evaluate the additional cost of the improved the methods, we count the number of function evaluations, additions/subtractions and multiplications/divisions to complete one step in each method.
For convenience, we present the procedures of each method here: \\
\\
Euler's method:
\begin{equation}
  y_j = y_{j-1} + f(x_{j-1}, y_{j-1})h
\end{equation}
The second-order Runga-Kutta method (RK2):
\begin{equation}
  k_1 = hf(x_n,y_n)
\end{equation}
\begin{equation}
  k_2 = h f(x_n + \frac{1}{2}h, y_n +\frac{1}{2}k_1)
\end{equation}
\begin{equation}
  y_{n+1} = y_n +k_2
\end{equation}
The quadratic approximation method:
\begin{equation}
  y_{j+1} = y_j + f(x_j, y_j)h +  \frac{\dd f}{\dd x}(x_j, y_j) \frac{h^2}{2}
\end{equation}
For the Euler's method, there is 1 addition, 1 multiplication and 1 function evaluation in each step; for the RK2 method, there
are 2 function evaluations, 3 additions and 4 multiplications in each step (we assume \(\frac{1}{2}\) is a known value, instead of 1 divided by 2); for the quadratic approximation method, there are 2 function evaluations (in the programming we usually precalculate \( \frac{\dd f}{\dd x}\) as a known function of \(x,y\)), 2 additions and 4 multiplications.
This information is summarized in the following table:
\begin{center}
\begin{tabular}{||c |c| c| c||}
 \hline
 & Euler's Method & RK2 & Quadratic Approximation \\ [0.5ex]
 \hline\hline
 the number of function evaluations,&&& \\
 additions/subtractions &&&\\
 and multiplications/divisions & 3& 9 & 8\\
  in each step  &  &  &  \\
 \hline
\end{tabular}
\captionof{table}{The number of function evaluations, additions/subtractions and multiplications/divisions to complete one step in each numerical method.}
\end{center}
Now, to a crude approximation, we assume each type of the operations counted here uses the same amount of time \(k\).
So, each step of Euler's method takes \(3k\), each step of the RK2 takes \(9k\) and each step of the quadratic approximation
takes \(8k\). Assume for a particular choice of \(N\), each method gives the same error \(E\). We wish to reduce the
error by a factor of \(2^{2n}\). For the Euler's method, we know it is first-order, which means
the error is proportional to the step size \(h\). If we reduce the error by a factor of \(2^{2n}\),
we need to reduce the step size \(h\) by a factor of \(2^{2n}\) as well. Since \(N \propto \frac{1}{h}\),
the number of steps \(N\) need to be increased by a factor of \(2^{2n}\), becoming \(2^{2n}N\).
For the RK2 and the quadratic approximation, since they are second order, the error is proportional to \(h^2\).
So, to reduce the error by a factor of \(2^{2n}\), we only need to reduce the step size \(h\) by a factor of \(2^n\).
Again, since  \(N \propto \frac{1}{h}\), the number of steps \(N\) need to be increased by a factor of \(2^{n}\), becoming \(2^{n}N\).
Now, the running time for each method is:
\begin{equation}
  t_{euler} = 3k 2^{2n}N
\end{equation}
\begin{equation}
  t_{rk2} = 9k 2^{n}N
\end{equation}
\begin{equation}
  t_{quad} = 8k 2^{n}N
\end{equation}
If \(n\) is large, \(t_{euler}\) would be much greater than \(t_{rk2}\) and \(t_{quad}\) (as \(3k\cdot 2^{n}\) would be much greater than \(9k\) and \(8k\)), even though
Euler's method takes less time in each step. So, the extra effort of the second-order methods definitely worth it!
\pagebreak
\section{Convergence and Stability}
The given differential equation is:
\begin{equation}
  y' + \alpha y =0
\end{equation}
where \(\alpha>0\) is a constant.
This is a separable differential equation. When \(y\ne 0\), we have:
\begin{equation}
   \int \frac{\dd y}{y} = -\alpha \int \dd x
\end{equation}
\begin{equation}
  \ln|y| = -\alpha x +C
\end{equation}
\begin{equation}
  y = \pm e^C e^{-\alpha x} = A e^{-\alpha x}
\end{equation}
where \(A\) is an arbitrary non-zero constant. We should also remark that \(y=0\) is a solution, as \((0)'+\alpha \cdot 0 =0\).
Including the zero solution, \(A\) can be any real number. So, the general solution is
\begin{equation}
   y = A e^{-\alpha x}
\end{equation}
We see that for any \(A\in \RR\),
\begin{equation}
  \lim_{x\to \infty} y = \lim_{x\to \infty} A e^{-\alpha x} = 0
\end{equation}
since \(\alpha >0\). This verifies all solutions of the given differential equation go to zero as \(x\to \infty\).
Now, suppose \(y_0, y_1, y_2, \ldots, y_N\) are approximate numerical solutions as \(x_0, x_1, \ldots,x_N=z\).
It is desirable that the numerical solutions have the same property as the analytical soltions \(y(x_n)\),
which go to zero as \(x_n \to \infty\). In the following section we will determine under what conditions
the Euler's method and the RK2 method have this property. \\
\subsection{Euler's Method}
For the Euler's method, we claim that
\begin{equation}
  y_n = (1-\alpha h)^n y_0
\end{equation}
for all \(n\in \NN\). We show this is true by induction.
\begin{proof}\text{ }\\
  When \(n = 0\), we have
  \begin{equation}
    y_0 = (1- \alpha h)^0 y_0
  \end{equation}
  This shows the above claim is true for \(n=0\). Now, assume it is true for \(n=k\), which means:
  \begin{equation}
    y_k = (1-\alpha h)^k  y_0
  \end{equation}
  We consider the case when \(n=k+1\). By the procedures of the Euler's method, we know that
  \begin{equation}
    y_{k+1} = y_k + f(x_k, y_k)h = y_k - \alpha h y_k  = y_k (1-\alpha h)
  \end{equation}
  Substitute the expression of \(y_k\) (the inductive hypothesis) into the above equation, we have:
  \begin{equation}
    y_{k+1} = (1-\alpha h) (1-\alpha h)^k  y_0 = (1-\alpha h)^{k+1}  y_0
  \end{equation}
  This shows the claim is true for \(n=k+1\). Now, we have shown that it is true for \(n=0\) and \(n=k+1\) assuming it is true
  for \(n=k\). By the principle of mathematical induction, we know that \(y_n = (1-\alpha h)^n y_0\) for all \(n\in \NN\).
\end{proof}\text{ }\\
The above result shows that \(y_n \to 0\) if and only if \(|1-\alpha h|<1\), \(0<\alpha h <2\).
Note that \(1/\alpha\) is the decay scale of the analytical solution. The result says that the step size must be
less than twice the decay scale for the solution to decay to zero at infinity.
\pagebreak
\subsection{The Second-order Runga-Kutta Method (RK2)}
For the RK2 method, we claim that:
\begin{equation}
  y_n = (1-\alpha h + \frac{(\alpha h)^2}{2})^n y_0
\end{equation}
for all \(n\in \NN\). We show this is true by induction.
\begin{proof}\text{ }\\
  When \(n=0\), we have:
  \begin{equation}
    y_0 = (1-\alpha h + \frac{(\alpha h)^2}{2})^0 y_0
  \end{equation}
  This shows the above claim is true for \(n=0\). Now, assume it is true for \(n=k\), which means:
  \begin{equation}
    y_k = (1-\alpha h + \frac{(\alpha h)^2}{2})^k y_0
  \end{equation}
  Consider the case when \(n = k+1\). By the procedures of the RK2 method, we have:
  \begin{equation}
    k_1 = f(x_k, y_k)h = -\alpha y_k h
  \end{equation}
  \begin{equation}
    k_2 = h f(x_k +\frac{1}{2}h, y_k + \frac{1}{2}k_1) = -h\alpha(y_k  - \frac{1}{2} \alpha y_k h) =y_k(-h\alpha +\frac{(\alpha h)^2}{2})
  \end{equation}
  \begin{equation}
    y_{k+1} = y_k + k_2 = y_k(1-h\alpha +\frac{(\alpha h)^2}{2})
  \end{equation}
  Substitute the expression of \(y_k\) (the inductive hypothesis) into the above equation, we have:
  \begin{equation}
    y_{k+1} = (1-h\alpha +\frac{(\alpha h)^2}{2}) (1-\alpha h + \frac{(\alpha h)^2}{2})^k y_0 = (1-\alpha h + \frac{(\alpha h)^2}{2})^{k+1} y_0
  \end{equation}
  This shows the claim is true for \(n=k+1\). Now, we have shown that the claim is true for \(n=0\) and \(n=k+1\)
  assuming it is true for \(n=k\). By the principle of mathematical induction, we know that
  \(  y_n = (1-\alpha h + \frac{(\alpha h)^2}{2})^n y_0\) for all \(n\in \NN\).
\end{proof}\text{ }\\
The above result shows that \(y_n\to 0\) if and only if \(|1-\alpha h + \frac{(\alpha h)^2}{2}|< 1\).
That is when \(-1<1-\alpha h + \frac{(\alpha h)^2}{2}<1\), \(\alpha h(\alpha h -2) < 0\), which is \(0<\alpha h <2\).
Again, the result says that the step size must be
less than twice the decay scale for the solution to decay to zero at infinity.
\end{document}
